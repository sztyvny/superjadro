%TODO:
%	-
%	-
\documentclass[a4paper,12pt]{article}


% to jest do polskich znakow
\usepackage[polish]{babel}
\usepackage[T1]{polski}
\usepackage[utf8]{inputenc} 
\usepackage{indentfirst}

\textheight=24.5cm
\textwidth=17cm
\topskip=5mm
\topmargin=-15mm
\leftmargin=-1mm
\oddsidemargin=-1mm      %10mm
\evensidemargin=-1mm     %10mm
\renewcommand{\baselinestretch}{1}

%najpierw sa informacje o dokumencie
%\title{Architektura Komputerow - Projekt}
%\author{Rafał Pieniążek \and Hubert Zapała}
%\date{28 maja}

\begin{document}
\thispagestyle{empty}

\noindent
\unitlength=1mm
\fboxrule=1mm
\setlength{\fboxrule}{3pt}\fbox{\LARGE\sf WT/P/17}\\[+2mm]
\begin{tabular}{ll}
\begin{minipage}[t]{110mm}
{\LARGE\sf 
Rafał Pieniążek\\
Hubert Zapała}
\end{minipage}
 &
\begin{minipage}[t]{55mm}
{\Large\sf 
\begin{tabular}{ll}
Ocena:  & \raisebox{-3mm}{\framebox(20,9)[cc]{}}\\
\end{tabular}
}
\end{minipage}
\end{tabular}

\hfill {\Large\sf Oddano: \raisebox{-3mm}{\framebox(30,9)[cc]{}}}\\[+10mm]

\begin{center}
{\huge\sf Jądro systemu operacyjnego}\\[+10mm]
{\Large\sc 
Architektura Komputerów 2 -- projekt\\
%Architektura komputer\'ow (2) -- projekt\\
2014/15\\[+10mm]}
\end{center}

\noindent
{\sc 
\hspace*{70mm}Prowadz\k{a}cy:\\
\hspace*{70mm}dr inż Tadeusz Tomczak\\
}

%\maketitle
\newpage
\tableofcontents % to jest do spisu treści
\newpage
	
	\section{Wstep}
		\subsection{Założenia projektowe i rzeczywiste osiągnięcia}
		W ramach projektu przygotowaliśmy podstawowe jądro systemu operacyjnego w trybie chronionym procesora. Jądro ma możliwość przełączania dwóch zadań za pomocą przerwań sprzętowych, pochodzących od klawiatury. Kod źródłowy jądra napisany jest w Turbo Assemblerze. Jądro to można uruchamiać w DOSBoxie, tak jak zwykły program, jednakże w ramach projektu zakupiliśmy stary komputer z procesorem Pentium III. Planowaliśmy uruchamiać system z dyskietki, ale nie poradziliśmy sobie z połączeniem jądra i bootloadera. Dlatego też jądro uruchamialiśmy z poziomu FreeDos, bez  wyższych sterowników grafiki. Kolejnymi etapami prac było:
			\begin{itemize}
				\item{instalacja środowiska}
				\item{kod przejścia procesora w tryb chroniony}
				\item{obsługa przerwań i wyjątków}
				\item{wykorzystanie pamieci rozszerzonej } (Aktywacja linii A20)
				\item{przełączanie zadań przez furtkę zadania }	
				\item{przełączanie zadań przez przerwania sprzętowe }	
			\end{itemize}
	\subsection{Wyjaśnienie pojęć}
		\begin{enumerate}
				\item{\textbf{segmentacja pamięci w trybie rzeczywistym} - adresy logiczne zawierają dwie składowe(adres bazowy segmentu oraz przesunięcie, czyli offset), które są tłumaczone na jednowymiarową pamięć fizyczną. W architekturze x86 wykorzystywane są 4 segmenty:}
					\begin{itemize}
					\item{programu(rejestr CS)}
					\item{stosu(rejestr(SS)}
					\item{danych(rejestr DS,ES)}
					\end{itemize}
				\item{\textbf{segmentacja pamięci w trybie chronionym} - w trybie tym istotne są rejestry sytemowe zawierające adresy bazowe tablic systemowych, służących do organizacji segmentacji. Rejestry te to:}
					\begin{itemize}
						\item{GDTR(Global Descriptor Table Register) - zawiera liniowy adres bazowy, oraz rozmiar globalnej tablicy deskryptorów}
						\item{IDTR(Interrupt Descriptor Table Register) - zawiera liniowy adres bazowy, oraz rozmiar tablicy deskryptorów przerwań}
						\item{LDTR(Local Descriptor Table Register) - zawiera selektor segmentu tablicy deskryptorów lokalnych}
						\item{TR(Task Register) - zawiera selektor segmentu stanu zadania}
					\end{itemize}
				\item {\textbf{deskryptor,tablice deskryptorów }- 8 bajtowa struktura danych określająca lokalizację segmentu programu w przestrzeni adresowej pamięci, oraz zasady dostępu do pamięci. Deskryptory są przechowywane w pamięci w postaci tablic deskryptorów.}
				\item{\textbf{selektor} jest 16 bitowym rejestrem zapisanym w jednym z rejestrów segmentowych. Jest on pewnego rodzaju wskaźnikiem na deskryptor w tablicy deskryptorów.}
					
				\item{\textbf{przerwanie} jest to zdarzenie, które prowadzi do obslugi przerwania. Źródłami przerwań mogą być podzespoły sprzętowe, wyjątki, lub instrukcje programowe. }
				%TUTAJ TRZEBA DOPISAĆ KILKA RZECZY, MOZE COS O ARCHITEKTURZE?
				%zadanie
				%TSS
			\end{enumerate}
	
	
	
	\subsection{Tryb chroniony}
	Procesory z rodziny x86  posiadają dodatkowy tryb , zwany trybem chronionym. Jest on szczególnie pomocny, podczas przełączania zadań. Podczas, gdy zadanie modyfikuje pewien obszar pamięci, musi być pewne, że wszystkie pozostałe zadania mają zablokowany dostęp do jego pamięci. W trybie chronionym ten problem został rozwiązany. Wszystkie zadania są w nim od siebie odseparowane, ponieważ dostęp do pamięci jest kontrolowany przez procesor. 

	\subsection{Przejście w tryb chroniony}
	Po włączeniu zasilania procesor jest w trybie rzeczywistym. Przejście w tryb chroniony odbywa się programowo, poprzez wykonanie pewnych instrukcji. Najpierw jednak należy zdefiniować strukturę - globalną tablicę deskryptorów - GDT.  Następnie należy ustawić bit PE(zezwolenie na tryb chroniony). Instrukcja LMSW pozwala nam pobrać całe słowo stanu procesora -  MSW. Po modyfikacji flagi, używamy WMSW, aby zapisać zmodyfikowane dane.
	
	\section{Jądro systemu w trybie chronionym}
	
	Podstawową funkcją jądra systemu operacyjnego jest możliwość przełączania zadań. Zadanie, jest to wykonywany program, lub niezależny jego fragment. Podczas przełączania zadań wykorzystywane są :
	\begin{itemize}
	\item{segment stanu zadania - TSS}
	\item{deskryptor segmentu stanu zadania}
	\item{rejestr zadania - TR}
	\item{deskryptor furtki zadania}
	\end{itemize}
Segment stanu zadania(\textit{z ang. Task State Segment}) jest to rekord, wchodzący w skład segmentu danych. 
%	\subsection{}
	\section{Zakonczenie}

	

\begin{thebibliography}{999}
\bibitem{aa1} W. Stanisławski, D.Raczyński 
{\em Programowanie systemowe mikroprocesorów rodziny x86},
PWN, Warszawa,2010.

\bibitem{aa2} G.Syck,
{\em Turbo Assembler - Biblia użytkownika}, 
LT\&P, Warszawa, 1994.

%\bibitem{aa3} A. Autor1, B. C. Autor2, 
%{\em Tytu\l{} ksi\k{a}\.zki}, 
%Nazwa Wydawcy, Miejsce Wydania, 1992.


\end{thebibliography}	

\end{document}

